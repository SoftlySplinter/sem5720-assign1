\documentclass[11pt, a4paper]{article}

\usepackage{amsmath}
\usepackage{amssymb}
\usepackage{graphicx}
\usepackage{listings}
\usepackage{color}
\usepackage[section]{placeins}
\usepackage{paralist}
\usepackage{fullpage}
\usepackage{glossaries}

\usepackage{caption}
\usepackage{subcaption}

\newacronym{ccTLD}{ccTLD}{country-code Top-Level Domain}
\newacronym{DNS}{DNS}{Domain Name System}
\newacronym{gTLD}{gTLD}{generic Top-Level Domain}
\newacronym{IANA}{IANA}{Internet Assigned Numbers Authority}
\newacronym{ICANN}{ICANN}{Internet Corporation for Assigned Names and Numbers}
\newacronym{IDN}{IDN}{Internationalized Domain Name}
\newacronym{ITLD}{ITLD}{Internationalized Top-Level Domain}
\newacronym{TLD}{TLD}{Top-Level Domain}
\newacronym{SEO}{SEO}{Search Engine Optimisation}

\newcommand*{\titleGM}{\begingroup
\hbox{ 
\hspace*{0.2\textwidth} 
\rule{1pt}{\textheight} 
\hspace*{0.05\textwidth} 
\parbox[b]{0.75\textwidth}{ 

{\noindent\Huge\bfseries Rapid Growth in Top Level Domains in the Domain Name System}\\[2\baselineskip] % Title
{\large \textit{SEM5720 - Assignment 1}}\\[4\baselineskip] % Tagline or further description
{\Large \textsc{Alexander D Brown (adb9)}} % Author name

\vspace{0.5\textheight} 
}}
\endgroup}


\begin{document}
\titleGM 
\tableofcontents
\newpage

\section{Introduction}

In recent years, the number of \gls{TLD} in the \gls{DNS} has been increasing.
This is, in part, due to the introduction of \gls{ICANN}, the aim of which was
to promote competition in the registration of domain names.

Before the creation of \gls{ICANN} in 1998 there were a total of eight 
\gls{gTLD}; \glspl{TLD} which are not specific to a country or the 
infrastructure of \gls{DNS}. These eight were intended to have specific uses
(\texttt{com} for commercial entities, \texttt{gov} for government 
organisations, etc.) but since then a proportion of these have become truly
generic.

In 2000 and 2004 \gls{ICANN} successfully applied for and instated fifteen new
\glspl{gTLD} and have since gone on to create a program which reviews all 
applications for new \glspl{gTLD}\cite{icann2013gtlds}.


\section{Expansion of TLDs}
At time of writing there are a total of 364 \glspl{TLD}, of which 295 are 
\glspl{ccTLD}, 42 \glspl{gTLD} (three of which are restricted use), 15 
sponsored \glspl{TLD}, 11 used for testing and 1 (\texttt{.arpa}) dedicated to 
the infrastructure for \gls{DNS}\cite{root2013iana}.

There are two factors which have spurred the growth of \glspl{TLD}:

\begin{enumerate}
\item Introduction of new \glspl{gTLD} by \gls{ICANN}
\item Support for non-Latin characters in the \gls{DNS}, allowing for 
      \glspl{ITLD}
\end{enumerate}


% Unicode in TLDs (Internationalised Domain Names).
Until October 2009\cite{icann2013idns} \glspl{TLD} could only consist of 
US-ASCII (or ``Latin'') characters. This changed with the approval of the
``IDN ccTLD Fast Track Process''\cite{icann2013final}, which allows countries 
to apply for \gls{IDN} \glspl{ccTLD} which represents their specific country 
(or territory) name in non-Latin script enabling users to access domain names
in their own language. %TODO this is almost too copypasta'd.

In more recent years, 2011 to be specific, \glspl{IDN} have also been approved 
for use in \glspl{gTLD}, allowing for a greater degree of freedom in 
international users.

% Increase in new TLDs (specifically gTLDs)
In hand with this there has been a fair increase in the number of \glspl{gTLD}
before 2010 there were only 22 \glspl{gTLD} available. On 12 January 2012,
applications for new \glspl{gTLD} opened and since then a total of 20 new 
\glspl{gTLD} have been added, including several \glspl{IDN}.

Even so, this is a drop in the ocean compared to the 1,930 applications 
\gls{ICANN} received and it is anticipated that the number of \glspl{gTLD} will
increase further in coming years.


\section{Non-technical Support of the Expansion of TLDs}

There is much support in favour of the expansion of \glspl{TLD}, much of which
is non-technical, especially in marketing and branding. The ability to have 
self-descriptive domain names is obviously appealing to businesses, especially
those with well known trademarks.

% Industry, product and service domains
% - self-descriptive domains.
Berneke\cite{berneke2013gtlds} gives ten reasons as to why \glspl{gTLD} are important
to businesses, explaining the notable factors such as the ability to have 
self-descriptive domains which clearly identify the area of a business.

There is the counter argument that \glspl{TLD} are becoming 
obsolete\cite{leiba2009good}, tools like search engines and increasingly
``intelligent'' web browsers are reducing the need for domain names. Though
this is obviously true, there is still a need for a human-readable name to 
access websites; the machine-readable IP addresses are not memorable, or short,
enough to easily distribute. Domain names also play a significant part in 
\gls{SEO}, although according to Google experts, the new \glspl{gTLD} will not
have any kind of preference over existing \glspl{TLD}\cite{cutts2012google}.

% Non-Latin Script (IDN) and Foreign Language Domains
% - China, Russia, etc.
Another important factor Berneke details the ability to target a international
demographic through the new non-Latin \glspl{IDN}. For those demographics which
use a non-Latin script, this is a vital element to reach the whole of the
target market.

% More choice
% - Initially only around 22 generic TLDs
% - Overcrowding of .net, .com, .org, etc.

% Localisation
% - .london

% Segment Focusing
% - Professionals (.doctor, .med, etc.)

% Increase in innovation and competition.

% Brand Protection


%\nocite{leiba2009good}
%\nocite{berneke2013gtlds}


\section{Technical Issues in the Expansion of TLDs}
% UDP jumbograms or TCP
% Firewalls
% Path MTUs

%\nocite{Manning2011Challenges}
%\nocite{kaufman2013unicode}

%\nocite{rfc3226}

\newpage
\bibliographystyle{IEEEtran}
\bibliography{citations}

\end{document}
