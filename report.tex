\documentclass[11pt, a4paper]{article}

\usepackage{amsmath}
\usepackage{amssymb}
\usepackage{graphicx}
\usepackage{listings}
\usepackage{color}
\usepackage[section]{placeins}
\usepackage{paralist}
\usepackage{fullpage}
\usepackage{glossaries}

\usepackage{caption}
\usepackage{subcaption}

\newacronym{ccTLD}{ccTLD}{country-code Top-Level Domain}
\newacronym{DNS}{DNS}{Domain Name System}
\newacronym{DNSSEC}{DNSSEC}{Domain Name System Security Extensions}
\newacronym{gTLD}{gTLD}{generic Top-Level Domain}
\newacronym{IANA}{IANA}{Internet Assigned Numbers Authority}
\newacronym{ICANN}{ICANN}{Internet Corporation for Assigned Names and Numbers}
\newacronym{IDN}{IDN}{Internationalized Domain Name}
\newacronym{ISP}{ISP}{Internet Service Provider}
\newacronym{ITLD}{ITLD}{Internationalized Top-Level Domain}
\newacronym{MTU}{MTU}{Maximum Transmission Unit}
\newacronym{TCP}{TCP}{Transmission Control Protocol}
\newacronym{TLD}{TLD}{Top-Level Domain}
\newacronym{TLS}{TLS}{Transport Layer Security}
\newacronym{TTL}{TTL}{Time To Live}
\newacronym{SEO}{SEO}{Search Engine Optimisation}
\newacronym{UDP}{UDP}{User Datagram Protocol}
\newacronym{URL}{URL}{Univerisal Resorce Locator}
\newacronym{QRCode}{QR Code}{Quick Resoponse Code}

\newcommand*{\titleGM}{\begingroup
\hbox{ 
\hspace*{0.2\textwidth} 
\rule{1pt}{\textheight} 
\hspace*{0.05\textwidth} 
\parbox[b]{0.75\textwidth}{ 

{\noindent\Huge\bfseries Rapid Growth in Top Level Domains in the Domain Name System}\\[2\baselineskip] % Title
{\large \textit{SEM5720 - Assignment 1}}\\[4\baselineskip] % Tagline or further description
{\Large \textsc{Alexander D Brown (adb9)}} % Author name

\vspace{0.5\textheight} 
}}
\endgroup}


\begin{document}
\titleGM 
\tableofcontents
\newpage

\section{Introduction}

In recent years, the number of \gls{TLD} in the \gls{DNS} has been increasing.
This is, in part, due to the introduction of \gls{ICANN}, the aim of which was
to promote competition in the registration of domain names.

Before the creation of \gls{ICANN} in 1998 there were a total of eight 
\gls{gTLD}; \glspl{TLD} which are not specific to a country or the 
infrastructure of \gls{DNS}. These eight were intended to have specific uses
(\texttt{com} for commercial entities, \texttt{gov} for government 
organisations, etc.) but since then a proportion of these have become truly
generic.

In 2000 and 2004 \gls{ICANN} successfully applied for and instated fifteen new
\glspl{gTLD} and have since gone on to create a program which reviews all 
applications for new \glspl{gTLD}\cite{icann2013gtlds}.


\newpage
\section{Expansion of TLDs}

At time of writing there are a total of 364 \glspl{TLD}, of which 295 are 
\glspl{ccTLD}, 42 \glspl{gTLD} (3 of which are restricted use), 15 
sponsored \glspl{TLD}, 11 used for testing and 1 (\texttt{.arpa}) dedicated to 
the infrastructure for \gls{DNS}\cite{root2013iana}.

There are two factors which have spurred the growth of \glspl{TLD}:

\begin{enumerate}
\item Introduction of new \glspl{gTLD} by \gls{ICANN}
\item Support for non-Latin characters in the \gls{DNS}, allowing for 
      \glspl{ITLD}
\end{enumerate}


% Unicode in TLDs (Internationalised Domain Names).
Until October 2009\cite{icann2013idns} \glspl{TLD} could only consist of 
US-ASCII (or ``Latin'') characters. This changed with the approval of the
``IDN ccTLD Fast Track Process''\cite{icann2013final}, which allows countries 
to apply for \gls{IDN} \glspl{ccTLD} which represents their specific country 
(or territory) name in non-Latin script enabling users to access domain names
in their own language. %TODO this is almost too copypasta'd.

In more recent years, 2011 to be specific, \glspl{IDN} have also been approved 
for use in \glspl{gTLD}, allowing for a greater degree of freedom in 
international users.

% Increase in new TLDs (specifically gTLDs)
In hand with this there has been a fair increase in the number of \glspl{gTLD}
before 2010 there were only 22 \glspl{gTLD} available. On 12 January 2012,
applications for new \glspl{gTLD} opened and since then a total of 20 new 
\glspl{gTLD} have been added, including several \glspl{IDN}.

Even so, this is a drop in the ocean compared to the 1,930 applications 
\gls{ICANN} received and it is anticipated that the number of \glspl{gTLD} will
increase further in coming years.


\newpage
\section{Non-technical Support of the Expansion of TLDs}

There is much support in favour of the expansion of \glspl{TLD}, much of which
is non-technical, especially in marketing and branding. The ability to have 
self-descriptive domain names is obviously appealing to businesses, especially
those with well known trademarks.

% Industry, product and service domains
% - self-descriptive domains.
Berneke\cite{berneke2013gtlds} gives ten reasons as to why \glspl{gTLD} are important
to businesses, explaining the notable factors such as the ability to have 
self-descriptive domains which clearly identify the area of a business.

There is the counter argument that \glspl{TLD} are becoming 
obsolete\cite{leiba2009good}, tools like search engines and increasingly
``intelligent'' web browsers are reducing the need for domain names. Though
this is obviously true, there is still a need for a human-readable name to 
access websites; the machine-readable IP addresses are not memorable, or short,
enough to easily distribute. Domain names also play a significant part in 
\gls{SEO}, although according to Google experts, the new \glspl{gTLD} will not
have any kind of preference over existing \glspl{TLD}\cite{cutts2012google}.

% Non-Latin Script (IDN) and Foreign Language Domains
% - China, Russia, etc.
Another important factor Berneke details the ability to target a international
demographic through the new non-Latin \glspl{IDN}. For those demographics which
use a non-Latin script, this is a vital element to reach the whole of the
target market.

% More choice
% - Initially only around 22 generic TLDs
% - Overcrowding of .net, .com, .org, etc.
With the addition of new \glspl{TLD} comes with a greater range of choice of
domain names. Before this there were only 22 \glspl{gTLD} and many of these are
restricted to specific institutions and organisations, leaving only a handful 
of domains remaining for non-\glspl{ccTLD} domains.

This has led to the overcrowding of the more popular \glspl{gTLD}; \texttt{.com},
\texttt{.org} and \texttt{.net} which has, in turn, ensured that desirable 
domains are either sparsely available or expensive to purchase. The arrival of
these new \glspl{gTLD} is expected to alleviate at least some of this 
overcrowding, but the author expects that the popular domains will still suffer
from this as entrepreneurial people snap up these domains at a cheap price and
sell them when the market prices are higher.

% Localisation
% - .london
A lot of major search engines are now placing increasing importance of 
geolocation\cite{linkdex2012georanking} and although a lot of this can be
expressed through \glspl{ccTLD} there are some proposed \glspl{gTLD} which
will encapsulate more specific geolocation through city information with 
\glspl{TLD} such as \texttt{.london} or \texttt{.nyc}. Of course the 
restrictions on this would have to be fairly tight and it may be almost 
impossible to police. Search engines are therefore likely to take the
\gls{TLD} information with a pinch of salt, but it may open up the market for
products relating to country information, cologne, for example, could be used
for the advertisement of fragrances.

% Segment Focusing
% - Professionals (.doctor, .med, etc.)
In hand with this, the ability to register a \gls{TLD} for a professional
career, a doctor for example, seems like a ideal way of stopping false 
advertising. These could prove to be easier to police, but again there is no
guarantee that a person owning a domain with the \gls{TLD} \texttt{.doctor} is
indeed a doctor.

For businesses that can help perform segment focusing, isolating the target
demographics to those \glspl{TLD} which appeal to them, be it art or football.

Even with this, there is already a lot of misinformation and ignorance as to
how the internet really works in the general population that these \glspl{gTLD}
may not be effective as hypothesised.

% Increase in innovation and competition.
% TODO If there's time and space

% Brand Protection
% TODO If there's time


\newpage
\section{Technical Issues in the Expansion of TLDs}

There have not been many technical issues with the root zone of \gls{DNS} until
fairly recently; two factors had helped in the stabilisation of 
this\cite{manning2011challenges}:

\begin{enumerate}
\item There were a fairly small number of \glspl{TLD}, allowing the size of the
      root to be around 80,000 bytes.
\item \glspl{TLD} were absorbed very slowly into the root zone and these
      changes were relatively small.
\end{enumerate}

Now, with the introduction of new \glspl{gTLD} and \glspl{ITLD}, there are 
going to be many more root zone entries and this increase is happening quite
quickly.

Not only are there going to be more \glspl{TLD} required, additional factors
like the new IPv6 protocol (discussed in subsection~\ref{subsec:dnsipv6}) and 
the need for security in \gls{DNS} through \gls{DNSSEC}\cite{rfc2535} 
(discussed in subsection~\ref{subsec:dnssec}). All these factors combined means
that the size of a \gls{DNS} query or response may not fit inside a 512 byte 
\gls{UDP} packet \gls{DNS} was initially intended to handle.

% UDP jumbograms or TCP
There are several options to cope with this; one approach would be to send a
\gls{UDP} packet greater than 512 bytes, another would be to use \gls{TCP} to
handle the sending of packets which are larger than 512 bytes. \gls{DNS} itself
is able to handle both of these options, but neither are particularly desirable
options.

Sending large \gls{UDP} packets over a network may run into the problem of 
requiring fragmentation if size of the path \gls{MTU} is less than the size of
the packet needing to be sent.

The other option, using \gls{TCP}, has a large amount of overhead involved, 
both in terms of packet size in adding the information required to perform a
connection in \gls{TCP} and also in terms of network traffic for the three-way 
handshake and for acknowledging received packets.

% Firewalls
Both these methods bring about problems when encountering firewalls. An 
incorrectly configured firewall might be set to not accept \gls{DNS} packets
with a size greater than 512 bytes via \gls{UDP} (the National Institute of Standards and 
Technology estimate that \gls{DNSSEC} will increase the \gls{DNS} response size
to be \textit{at least} 2048 bytes), so a response from a \gls{DNSSEC} signed 
response would be dropped by the firewall.

An incorrectly firewall would likely drop \gls{DNS} traffic sent using
\gls{TCP}, the alternative for sending larger \gls{DNS} messages. This means
that the potential worst-case scenario is that \gls{DNS} traffic signed using
\gls{DNSSEC} would not be resolved at all and it would fall back to using 
unsigned \gls{DNS} messages, but having caused a lot of traffic on the network.


% Path MTUs
The other problem comes with the path \gls{MTU} size. The guidelines for using
\gls{DNSSEC} state that the client must be able to accept \gls{DNS} messages of
at least 2048 bytes, the \gls{MTU} size of the path these packets would have to
traverse may be limited to a smaller amount, requiring the packets to be 
fragmented, which may lead to data loss and redundant retransmission, or for 
\gls{TCP} to be considered, with all of its overheads.

This may become less of a problem as the \gls{MTU} size increases as networks
are improved with modern technology, such as fibre optics. But the path 
\gls{MTU} is limited by the weakest link in the chain so this may remain a
problem for many years to come.

\subsection{DNS Security}
\label{subsec:dnssec}
% Explain DNSSEC
In recent years, the need for security in the \gls{DNS} has become apparent, 
simple attacks like \gls{DNS} Spoofing, where the response from a \gls{DNS}
query is not for the correct server, but instead for a different server, 
typically the attackers. One method of doing this is \gls{DNS} Cache 
Poisoning\cite{davies2008cache}, where the principal of saving lookup time by
caching results leads to the wrong result being cached for as long as the 
\gls{TTL} of the cache. 

This can even affect servers from the root zones when a
name server provides both an authoritative and recursive name service, where an
attack on the recursive side would lead to bad data given to computers wanting 
an authoritative answer; the net result of which is that one could insert or
modify domain data inside a \gls{TLD}. %TODO way too copypasta'd

The formal solution to this is to introduce security to the \gls{DNS} through
public key cryptography to sign the responses given by the recursive \gls{DNS}
lookup, allowing the response to be verified by the client. \Gls{DNSSEC} is the
mechanism used to perform this process.

The main issue with \gls{DNSSEC} is that it is a complex system which requires
a decent level of expertise to set up and is only applicable to a fairly small
proportion of businesses. This, combined with a lack of application support for
\gls{DNSSEC} means that, at the moment at least, the reward for implementing
\gls{DNSSEC} is often not worth the risk\cite{rasmussen2011risk}.

\subsection{DNS Support for IPv6}
\label{subsec:dnsipv6}
% Explain IPv6

The \gls{DNS} cannot be easily extended to support IPv6 addresses, since it is
assumed by clients that a 32-bit IPv4 address will be returned and not a 128-bit
IPv6 address.

A new resource record was implemented in \gls{DNS} to accommodate IPv6 
addresses, the AAAA record, by RFC3596\cite{rfc3596}.

The main issue brought about by these changes in \gls{DNS} is the increase of
size in the query/response messages due to the four times increase in number of
bit required to represent the IP address.

% Problems with Unicode
\subsection{IDNs and Issues with Unicode}
There is also the question of the safety of using a non-Latin encoding
(e.g. Unicode) where certain characters may look either very similar or even
indistinguishable to one another\cite{kaufman2013unicode}. This can lead to the
spoofing of known domain names which attackers could use for malicious purposes.

Of course, the registration of \glspl{IDN} at the root zone is restricted to
prevent this\cite{icann2011idns}, but that has not meant that the implementation of \glspl{IDN} has
not caused issues with other areas involving non-Latin encoding, The Homograph 
Attack\cite{gabrilovich2002homograph}, for example.


\newpage
\section{Evaluation}

It is obvious that the need for new \glspl{TLD} is an important step; however, 
it seems like a step that has taken far too long to put into place. A lot of
arguments are now made that the use of search engines has deprecated the need
for a truly memorable \gls{URL}, let alone a \gls{TLD} and whilst the reserved
\gls{TLD} for government and educational purposes do make these institutions
easily recognisable and verifiable, the levels of information the proposed 
\glspl{gTLD} could offer seem too detailed and too difficult to police for the
little information they would provide to the user.

Another argument one could make against the new \glspl{TLD} is that sites 
providing \gls{URL} shortening or technologies such as \glspl{QRCode} are now
more popular methods of sharing location information more memorably and would
seem to have sprung up as a way around the lack of domains available.

The need for technologies like \gls{DNSSEC} should be obvious to all users, but
when different technologies like \gls{TLS} also provide a similar service, with
a lot of added benefits, publicity and only when required, one does wonder 
whether the need for it is quite so pressing, especially given the overhead that
would be added by \gls{DNSSEC} especially if a lot of users have incorrectly
configured firewalls, ones on a router provided by their \glspl{ISP} for 
example.

Of course, \gls{DNSSEC} is at least an automatic process where domain names 
which do not match the signature would be heralded with a large warning from
the browser, whilst technologies like \gls{TLS} are not guaranteed to, or might
be used where users might assume it is (e.g. spoof websites using HTTP rather 
than HTTPS would not show any warning).

\gls{DNS} support for IPv6 is already fairly good and even though it does 
increase the packet size, it is necessary step to help the widespread
adoption of IPv6. One might tentatively predict that the packet size might be
able to become somewhat smaller once IPv4 becomes less prevalent, but it would
undoubtedly be many years before this happens.

\newpage
\bibliographystyle{IEEEtran}
\bibliography{citations}

\end{document}
