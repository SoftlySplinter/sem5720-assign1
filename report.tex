\documentclass[11pt, a4paper]{article}

\usepackage{amsmath}
\usepackage{amssymb}
\usepackage{graphicx}
\usepackage{listings}
\usepackage{color}
\usepackage[section]{placeins}
\usepackage{paralist}
\usepackage{fullpage}
\usepackage{glossaries}

\usepackage{caption}
\usepackage{subcaption}

\newacronym{DNS}{DNS}{Domain Name System}
\newacronym{IANA}{IANA}{Internet Assigned Numbers Authority}
\newacronym{ICANN}{ICANN}{Internet Corporation for Assigned Names and Numbers}
\newacronym{TLD}{TLD}{Top-Level Domain}
\newacronym{gTLD}{gTLD}{generic Top-Level Domain}
\newacronym{ccTLD}{ccTLD}{country-code Top-Level Domain}
\newacronym{ITLD}{ITLD}{Internationalized Top-Level Domain}

\newcommand*{\titleGM}{\begingroup
\hbox{ 
\hspace*{0.2\textwidth} 
\rule{1pt}{\textheight} 
\hspace*{0.05\textwidth} 
\parbox[b]{0.75\textwidth}{ 

{\noindent\Huge\bfseries Rapid Growth in Top Level Domains in the Domain Name System}\\[2\baselineskip] % Title
{\large \textit{SEM5720 - Assignment 1}}\\[4\baselineskip] % Tagline or further description
{\Large \textsc{Alexander D Brown (adb9)}} % Author name

\vspace{0.5\textheight} 
}}
\endgroup}


\begin{document}
\titleGM 
\tableofcontents
\newpage

\section{Introduction}

In recent years, the number of \gls{TLD} in the \gls{DNS} has been increasing.
This is, in part, due to the introduction of \gls{ICANN}, the aim of which was
to promote competition in the registration of domain names.

Before the creation of \gls{ICANN} in 1998 there were a total of eight 
\gls{gTLD}; \glspl{TLD} which are not specific to a country or the 
infrastructure of \gls{DNS}. These eight were intended to have specific uses
(\texttt{com} for commercial entities, \texttt{gov} for government 
organisations, etc.) but since then a proportion of these have become truly
generic.

In 2000 and 2004 \gls{ICANN} successfully applied for and instated fifteen new
\glspl{gTLD} and have since gone on to create a program which reviews all 
applications for new \glspl{gTLD}\cite{icann2013gtlds}.


\section{Expansion of TLDs}
At time of writing there are a total of 363 \glspl{TLD}, of which 295 are 
\glspl{ccTLD}, 42 \glspl{gTLD} (three of which are restricted use), 15 sponsored
\glspl{TLD} and 11 which are used for testing purposes\cite{root2013iana}. 

There are a few factors which have spurred the growth of \glspl{TLD}:

\begin{enumerate}
\item Introduction of new \glspl{gTLD} by \gls{ICANN}
\item Support for non-Latin characters in the \gls{DNS}, allowing for 
      \glspl{ITLD}
\end{enumerate}
% Increase in new TLDs (specifically gTLDs)
% Unicode in TLDs (Internationalised Domain Names).


\section{Non-technical Support of the Expansion of TLDs}
% Industry, product and service domains
% - self-descriptive domains.

% Non-Latin Script (IDN) and Foreign Language Domains
% - China, Russia, etc.

% More choice
% - Initially only around 22 generic TLDs
% - Overcrowding of .net, .com, .org, etc.

% Localisation
% - .london

% Segment Focusing
% - Professionals (.doctor, .med, etc.)

% Increase in innovation and competition.

% Brand Protection




\section{Technical Issues in the Expansion of TLDs}
% UDP jumbograms or TCP
% Firewalls
% Path MTUs


\nocite{Manning2011Challenges}
\nocite{rfc3226}
\nocite{root2013iana}
\nocite{berneke2013gtlds}
\nocite{kaufman2013unicode}
\nocite{leiba2009good}

\newpage
\bibliographystyle{IEEEtran}
\bibliography{citations}

\end{document}
